\documentclass[modern]{aastex63}

\usepackage[utf8]{inputenc}
\usepackage{amsmath}

\newcommand{\SBUAffil}{Stony Brook University, Stony Brook, NY 11794}

\DeclareMathOperator{\var}{Var}

\begin{document}
\title{Determining the Hubble Constant from Observations of Distance Modulus and
Redshift for Type Ia Supernovae}

% -- List each individual author --
\author{Henry Shi}
\email{henry.shi@stonybrook.edu}
\affiliation{\SBUAffil}

\author{Will Farr, Ph.D.}
\email{will.farr@stonybrook.edu}
\affiliation{\SBUAffil}

\section{Abstract}

We present the distance moduli and redshifts for 1048 confirmed Type Ia
supernovae (SNe Ia) from Scolnic 2018. These SNe Ia range from
$0.01 < z < 2.3$ and are compiled from the Pan-STARRS1 (PS1) Medium Deep Survey,
Sloan Digital Sky Survey (SDSS), SNLS, low-z, and Hubble Space Telescope
observations. We call this dataset the "Pantheon sample". By fitting a model of
distance modulus vs redshift to the data, we obtain values of three parameters
that shed light on the universe: mass density $\Omega_m$, equation of state $w$,
and distance modulus offset $\Delta_{dm}$ of the model from the data. Using the
Markov-Chain Monte Carlo sampling method for 10,000 iterations yields
$\Omega_m=0.337\pm0.066$, $w=-1.182\pm0.521$, $\Delta_{dm}=10.634\pm0.261$.
From these parameters we hope to determine the
value of the Hubble constant, $H_0$. $H_0$ is related to the speed of expansion
of the universe via the equation $v = H_0 d$, where $v$ the the velocity of a
receding galaxy and $d$ is the galaxy's distance from Earth. With these results,
our experiment fails to reproduce the results of Scolnic 2018.
In addition, we are unable to separate random error
and systematic error in our parameter calculations.
Further refinement of our sampling algorithm and further processing of
$\Omega_m$, $w$, and $\Delta_{dm}$ are needed to determine the Hubble constant
$H_0$. By obtaining a precise estimate of the Hubble constant, we
hope to gain insight into the history of the universe.

\textit{Key words:} cosmology, model fitting, Markov Chain sampling, Type Ia supernova,
distance modulus, Hubble constant

\textit{Sources:}\newline
Scolnic, D.M. \textit{The Astrophysical Journal}, 859:101 (2018).
\newline
Hogg, David W. "Data Analysis Recipes: Using Markov Chain Monte Carlo" (2017). arXiv:1710.06068v1.
\end{document}
